\chapter*{\hfill{\centering ЗАКЛЮЧЕНИЕ}\hfill}
\addcontentsline{toc}{chapter}{ЗАКЛЮЧЕНИЕ}

В результате выполнения данного курсового проекта проведен анализ методов представления объектов, алгоритмов удаления невидимых линий и поверхностей, а также моделей освещения. 
В работе указаны преимущества и недостатки этих методов.

На основе результатов анализа разработано программное обеспечение, которое позволяет создавать собственные сцены, используя сферы, параллелепипеды, кубы и пирамиды. 
В результате проектирования были созданы новые и адаптированы существующие структуры данных и алгоритмы для решения задачи.

Эксперименты по сравнению реализаций трассировки лучей выявили, что для ускорения процесса конструирования сцены не обязательно использовать реалистичные изображения, что позволяет упростить алгоритм трассировки лучей.

В дальнейшем развитии программы можно рассмотреть расширение ее функционала для взаимодействия не только с примитивами, но и с более сложными объектами, созданными другими программами. 
Это позволит конструировать более сложные сцены и получать их более реалистичные изображения.

В результате выполнения курсовой работы были выполнены следующие задачи:
\begin{enumerate}[label={\arabic*)}]
	\item проанализированы имеющиеся алгоритмы и определены оптимальные методы для решения основной задачи;
	\item разработана эффективная структура программы;
	\item выбран наиболее подходящий язык программирования и интегрированную среду разработки для выполнения задачи;
	\item создан программны продукт для решения задачи и реализованы выбранные алгоритмы;
	\item создан понятный пользовательский интерфейс;
	\item проведены исследования, основанные на полученных результатах.
\end{enumerate}