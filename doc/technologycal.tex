\chapter{Технологическая часть}

В этом разделе осуществляется выбор способов реализации программного обеспечения (ПО), также предоставляются реялизации выполенных алгоритмов. 

\section{Выбор и обоснование языка программирования и среды разработки}

В качестве языка программирования был выбран \texttt{C\#}, так как:
\begin{itemize}
	\item я ознакомился с этим языком программирования во время обучения, что сократит время написания программы;
	\item этот язык программирования объектно-ориентирован, что позволяет:
	\begin{itemize}
		\item использовать наследование, абстрактные классы и т.д.
		\item представлять трехмерные объекты сцены в виде объектов классов, что позволяет легко организовать взаимодействие между ними, положительно влияя на читабильность, не снижая эффективности.
	\end{itemize}
\end{itemize}

В качестве среды разработки был выбран Visual Studio, так как:
\begin{itemize}
	\item он бесплатен в использовании студентам;
	\item я знаком с данной средой разработки, что сократить время изучения возможностей среды.
\end{itemize}
\clearpage

\section{Реализация алгоритмов}

В листингах 3.1 -- 3.3 представлены реализации рассматриваемых алгоритмов.

\bigskip
\begin{lstlisting}[caption=Простой алгоритм обратной трассировки лучей]
	public Vector3 TraceRay(Vector3 camera, Vector3 D, double t_min, double t_max, int x, int y)
	{
		double closest_t = Double.PositiveInfinity;
		Primitive closest_object = null;
		
		ClosestIntersection(ref closest_object, ref closest_t, camera, D, t_min, t_max);
		
		if (closest_object == null)
			return new Vector3();
		
		
		Vector3 P = camera + closest_t * D; 
		Vector3 N;
		
		if (closest_object is Parallelepiped)
			N = Vec3dNormalParallelepiped(P, (Parallelepiped)closest_object);
		else if (closest_object is Cube)
			N = Vec3dNormalCube(P, (Cube)closest_object);
		else if (closest_object is Triangle)
			N = Vec3dNormalTriangle((Triangle)closest_object);
		else
			N = P - closest_object.position; 
		N = N / Vector3.Length(N);
		
		double intensity = ComputeLighting(P, N, -D, closest_object.specular);
		
		return intensity * closest_object.color;
	}
\end{lstlisting}
\clearpage

\begin{lstlisting}[caption=Модифицированный алгоритм обратной трассировки лучей]
	private Vector3 TraceRay(Vector3 camera, Vector3 dir, double t_min, double t_max, int depth)
	{
		double closest_t = Double.PositiveInfinity;
		Primitive closest_object = null;
		
		ClosestIntersection(ref closest_object, ref closest_t, camera, dir, t_min, t_max);
		
		if (closest_object == null)
			return new Vector3();
		
		Vector3 P = camera + closest_t * dir;
		Vector3 N;
		if (closest_object is Parallelepiped)
			N = Vec3dNormalParallelepiped(P, (Parallelepiped)closest_object);
		else if (closest_object is Cube)
			N = Vec3dNormalCube(P, (Cube)closest_object);
		else if (closest_object is Triangle)
			N = Vec3dNormalTriangle((Triangle)closest_object);
		else
			N = P - closest_object.position;
		N = N / Vector3.Length(N);
		
		double intensity = ComputeLighting(P, N, -dir, closest_object.specular);
		
		Vector3 localColor = intensity * closest_object.color;
		
		double r = closest_object.reflective;
		
		if (depth <= 0 || r <= 0)
			return localColor;
		
		Vector3 R = ReflectRay(-dir, N);
		Vector3 reflectedColor = TraceRay(P, R, 0.001, Double.PositiveInfinity, depth - 1);
		
		Vector3 kLocalColor = (1 - r) * localColor;
		Vector3 rReflectedColor = r * reflectedColor;
		
		return kLocalColor + rReflectedColor;
	}
\end{lstlisting}
\clearpage

\begin{lstlisting}[caption=Алгоритм Фонга]
private double ComputeLighting(Vector3 P, Vector3 N, Vector3 V, double specular)
{
	double intensity = scene.ambient_light.intensity;
	List<LightSource> sceneLight = scene.light_sources;
	
	for (int i = 0; i < sceneLight.Count; i++)
	{
		Vector3 L;
		double t_max;
		L = sceneLight[i].position - P;
		t_max = Double.PositiveInfinity;
		
		double shadow_t = Double.PositiveInfinity;
		Primitive shadow_object = null;
		ClosestIntersection(ref shadow_object, ref shadow_t, P, L, 0.001, t_max);
		if (shadow_object != null)
			continue;
		
		double n_dot_l = Vector3.ScalarMultiplication(N, L);
		
		if (n_dot_l > 0)
		{
			intensity += sceneLight[i].intensity * n_dot_l / (Vector3.Length(N) * Vector3.Length(L));
		}
		
		if (specular != -1)
		{
			
			Vector3 R = 2 * N * n_dot_l - L;
			double r_dot_v = Vector3.ScalarMultiplication(R, V);
			
			if (r_dot_v > 0)
			{
				intensity += sceneLight[i].intensity * Math.Pow(r_dot_v / (Vector3.Length(R) * Vector3.Length(V)), specular);
			}
		}
	}
	return intensity;
}
\end{lstlisting}
\clearpage


