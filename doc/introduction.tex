\chapter*{\hfill{\centering ВВЕДЕНИЕ}\hfill}
\addcontentsline{toc}{chapter}{ВВЕДЕНИЕ}

В области компьютерной графики, значительное внимание уделяется разработке алгоритмов для создания более реалистичных изображений. 
Однако, по мере усложнения самих алгоритмов, возрастают требования к вычислительным ресурсам системы. 
Эта исследовательская работа актуальна, поскольку она стремится найти оптимальные алгоритмы, которые могут быть использованы для аппаратного создания реалистичных изображений, с учетом ограниченных ресурсов компьютерной системы.

Цель курсовой работы: разработать программу моделирования блочного конструктора из перечня геометрических объектов: куб, шар, параллелепипед, наклонная усеченая пирамида.

Для достижения цели, требуется выполнить следующие
задачи:
\begin{enumerate}[label={\arabic*)}]
	\item проанализировать имеющиеся алгоритмы и определить оптимальные методы для решения основной задачи;
	\item разработать эффективную структуру программы;
	\item выбрать наиболее подходящий язык программирования и интегрированную среду разработки для выполнения задачи;
	\item создать программный продукт для решения задачи, реализовать выбранные алгоритмы;
	\item создать понятный пользовательский интерфейс;
	\item провести исследования, основанные на полученных результатах.
\end{enumerate}